\documentclass[a4paper,10pt]{article}
\usepackage{multirow}
\usepackage{graphicx}
\usepackage{fancyhdr}
\usepackage{lastpage}
\usepackage{amssymb}
\usepackage{epsfig}
\usepackage{latexsym}
\usepackage{amstext}
\usepackage{amsmath}
\usepackage{amsfonts}
\usepackage{amsthm}
\usepackage{array}
\usepackage{tikz}
\usepackage{fancyvrb}
\usepackage[a4paper, total={6.5in, 10in}]{geometry}
\usepackage{gfsartemisia-euler}
\usepackage[T1]{fontenc}
\usepackage[latin1]{inputenc}
\usepackage[spanish]{babel}

\newtheorem{teo}{Teorema}

\newcommand{\Mc}[1]{\ensuremath{M_{91}(#1)}}
\newcommand{\Mci}[2]{\ensuremath{M_{91}^{#2}(#1)}}
\newcommand{\Mi}[2]{\ensuremath{M^{#2}(#1)}}
\newcommand{\Ind}[1]{\ensuremath{\mathbb{I}\{#1\}}}
\newcommand{\Nat}{\ensuremath{\mathbb{N}}}
\newcommand{\ceil}[1]{\ensuremath{\left\lceil #1 \right\rceil}}


\begin{document}
\pagestyle{fancyplain}
\cfoot{- \thepage/\pageref{LastPage} -}
\lhead{}
\chead{}
\rhead{}
\renewcommand{\headrulewidth}{0pt}

\section{Introducci\'on}

En este documento pretendemos analizar la funci\'on 91 de McCarthy a trav\'es de una generalizaci\'on
de la misma. La idea de este enfoque es abstraer las constantes por variables y luego razonar
anal\'iticamente sobre esta funci\'on generalizada con el objeto de poder encontrar expresiones
cerradas para el valor de la funci\'on y para la cantidad de llamadas recursivas efectuadas para un
argumento dado.

\section{La funci\'on 91 de McCarthy}

La funci\'on 91 de McCarthy se define matem\'aticamente como sigue:

$$
\Mc{n} =
    \begin{cases}
        n - 10            & \text{si } n > 100 \\
        \Mc{\Mc{n+11}}  & \text{en otro caso}
    \end{cases}
$$

Otra forma alternativa (y m\'as concisa) de expresarla es as\'i:

$$\Mc{n} = (n - 10) \cdot \Ind{n > 100} + \Mci{n+11}{2} \cdot \Ind{n \leq 100}$$

En esta forma,
\begin{itemize}
    \item $\Ind{p}$ es un indicador booleano que vale 1 si y s\'olo si el predicado $p$ es verdadero, y
    \item $\Mci{n}{i} = \underbrace{M_{91}(M_{91}(\dots \Mc{n}}_{i \text{ veces}}\dots))$.
\end{itemize}

\subsection{Generalizaci\'on}

Podemos obtener una versi\'on generalizada $M = M_{m,k,s,e}$ de la funci\'on 91 de McCarthy abstrayendo cada constante en
una variable distinta:

$$M(n) = (n - s) \cdot \Ind{n > m} + \Mi{n+k}{e} \cdot \Ind{n \leq m}$$

Es claro entonces que $M_{91}$ es un caso particular de $M$, dado que podemos generarla mediante la instanciaci\'on
$M_{91} = M_{100,11,10,2}$.

\section{An\'alisis}

\subsection{La relaci\'on entre $k$, $s$ y $e$}

El primer interrogante que surge es bajo qu\'e condiciones la funci\'on $M$ est\'a bien definida. Esto es, c\'omo debe
ser la relaci\'on entre las distintas variables de forma tal que $M(n)$ tenga un valor num\'erico y no haya divergencia
computacional al intentar calcular dicho valor (hecho que notaremos con el s\'imbolo $\bot$).
Para ello probaremos el siguiente resultado:

\begin{teo}
Si $k \leq (e - 1) \cdot s$, se tiene que $M(n) = \bot$ para cualquier $n \leq m$.
\end{teo}

\begin{proof}
Sea $n \leq m$. Debe existir entonces un $c \in \Nat$ tal que 

$$M(n) = \Mi{m - k + i}{c}$$

con $1 \leq i \leq k$. Esto corresponde al hecho sumar sucesivamente $k$ al argumento hasta alcanzar el m\'aximo
valor posible antes de hacer el primer decremento por $s$. Desplegando la invocaci\'on m\'as interna, que 
representa sumar una vez m\'as $k$ y agregar otras $e$ llamadas, se tiene

$$\Mi{m - k + i}{c} = \Mi{m + i}{c + e - 1}$$

Sea $e_0 = \text{m\'in} \, \left\{ a \in \Nat \, / \, m + i - as \leq m \right\}$. Entonces,

\begin{eqnarray*}
\Mi{m - k + i}{c} &=& \Mi{m + i}{c + e - 1} \\
                  &=& \Mi{m + i - e_0 s}{c + e - 1 - e_0}
\end{eqnarray*}
Como sabemos que $k \leq (e-1) \cdot s$, $m + i \leq m + k \leq m + (e-1) \cdot s$. Substrayendo $(e-1)\cdot s$ de
ambos extremos, tenemos que $m + i - (e-1)\cdot s \leq m$. De esto sigue que $e_0 \leq e-1$, pues por
definici\'on es el menor n\'umero natural con dichas propiedades. Luego,

$$c + e - 1 - e_0 \geq c + e - 1 - (e-1) = c$$

El mismo razonamiento aplica para el argumento $m + i - e_0 s \leq m$. De esto se concluye que la cantidad de 
llamadas a $M$ es creciente, con lo cual $M(n) = \bot$.
\end{proof}

La interpretaci\'on fundamental de este resultado es que $k$ debe valer por lo menos $(e-1)\cdot s + 1$ para que
$M$ est\'e bien definida. En el caso de $M_{91}$, observamos que $k$ toma el m\'inimo valor posible, pues
$k = 11 = 1 \cdot 10 + 1 = (e - 1) \cdot s + 1$.

\subsection{C\'alculo del valor de la funci\'on}

Tal como puede verse, la funci\'on $M$ anida llamadas en las que incrementa una y otra vez en $k$ el argumento, para
eventualmente decrementarlo por $s$ cada vez que se supera el umbral $m$. Por este motivo, se puede concluir que

$$M(n) = n + \alpha_n k - \beta_n s$$

para algunos $\alpha_n, \beta_n \in \Nat$. Ahora bien, de cada $e$ llamadas, una de ellas hace el incremento en $k$. Esto
implica que $\alpha_n = r(n) / e$, siendo $r(n)$ la cantidad de llamadas recursivas efectuadas al computar
$M(n)$. Naturalmente, $r(n)$ debe ser m\'ulitplo de $e$ pues cada vez que se anidan llamadas recursivas, la cantidad
anidada es $e$. Por otro lado, toda otra llamada que no incremente el argumento en $k$ debe resolverse
por decremento en $s$ eventualmente, dado que no existe otra posibilidad. Luego, $\alpha_n = 1 + r(n) - r(n)/e$ (se
suma 1 pues hay que contabilizar tambi\'en la llamada original a $M$). Entonces,

$$M(n) = n + \frac{r(n)}{e} \cdot k - \left(1 + \frac{(e-1) \cdot r(n)}{e}\right) \cdot s$$

En consecuencia, el c\'omputo de $M(n)$ se reduce a calcular la cantidad de llamadas recursivas $r(n)$. 

\subsection{C\'alculo de llamadas recursivas cuando $e = 2$}

Calcular $r(n)$ en el caso m\'as general parece ser dif\'icil. No obstante, vamos a dar un argumento constructivo
para calcular este valor cuando $e = 2$ y cuando $k = s + 1$. A continuaci\'on, generalizaremos la expresi\'on
encontrada para abarcar valores de $k$ arbitrarios, y probaremos que dicha f\'ormula es correcta.

\subsubsection{El caso f\'acil: $k = s + 1$}

En este escenario, la funci\'on $M$ sigue un patr\'on de incremento lineal en la cantidad de llamadas recursivas a
resolver, y en cada uno de estos pasos se procede sumando $k$ al argumento:

$$M(n) = M^2(n + k) = M^3(n + 2k) = \cdots = M^{i+1}(n + ik)$$

Este patr\'on termina cuando $n + ik > m$, lo que equivale a decir que $i$ es el m\'inimo n\'umero entero mayor
que $(m - n) / k$. En otras palabras,

$$i = \ceil{\frac{m-n+1}{k}}$$

Entonces,

\begin{eqnarray*}
M(n)  &=& M^{i+1}(n + ik) \\
      &=& M^i(n_0 = n + ik - s) \\
      &=& M^i(n_1 = n + ik - s + (k-s)) \\
      &\vdots& \\
      &=& M^i(n_j = n + ik - s + j(k-s))
\end{eqnarray*}

en donde $n_0,\cdots,n_j \leq m$.

Como $k = s + 1$, tenemos que $n_l = 1 + n_{l-1}$ (i.e., el argumento de $M^i$ se incrementar\'a
de a 1). Pero esto s\'olo se da hasta llegar a $n_j = m$, en cuyo caso aparece un $s$ adicional restando (pues
$m + k - s = m + 1 > m$):

\begin{eqnarray*}
M(n) &=& M^i(n_j = m = n + ik - s + j(k-s)) \\
     &=& M^{i-1}(n + ik - s + (j+1)(k-s) - s = m + 1 - s)
\end{eqnarray*}

En este punto, se observa que se decrement\'o en 1 el exponente de M, lo cual implica que tenemos una llamada
anidada menos a resolver. Este patr\'on de subir de a 1 el argumento hasta alcanzar a $m$ y luego sacar una
llamada anidada va a seguir d\'andose hasta que eventualmente el exponente llegue a 0.\\

A partir de este desarrollo, el c\'alculo de $r(n)$ lo podemos realizar por partes:

\begin{itemize}
  \item Por un lado, contar las llamadas que tenemos desde el principio hasta que alcanzar a $n_0$ (i.e.,
        hasta que llegamos a superar a $m$). Esto not\'emoslo $r_0(n)$.
  \item Despu\'es, contar las llamadas que tenemos en el incremento de $n_0$ hasta llegar a $m$ (not\'emoslo $r_1(n)$).
  \item Luego, las llamadas involucradas en el incremento del par\'ametro, partiendo desde $m + 1 - s$, hasta bajar
        en 1 el exponente (a lo que notaremos $r_2$, y no depende de $n$).
  \item Una vez calculado esto, tenemos que $r(n) = r_0(n) + r_1(n) + (i-1) \cdot r_2$, dado que como vimos el ciclo
        de incremento/decremento se da hasta agotar el exponente de M.
\end{itemize}

Tenemos que $r_0(n) = 2i$: por cada llamada a $M$, aparecen otras dos. Adem\'as,
$$r_1(n) = 2  \cdot  \text{cantidad de n\'umeros entre} (n+ik-s) \text{ y } m$$

dado que son las llamadas que tenemos hasta llegar por primera vez a $m$ para decremenetar el exponente. Entonces,

$$r_1(n) = 2 \cdot (m  - (n + ik - s) + 1) = 2 \cdot (m - n - ik + s + 1)$$

Finalmente, $r_2 = 2s$, dado que cada ciclo tiene $s$ n\'umeros: desde $m+1-s$ hasta $m$. Juntando todo,

\begin{eqnarray*}
r(n) &=& 2i + 2 \cdot (m - n - ik + s + 1) + 2 \cdot (i-1) \cdot s \\
     &=& 2 \cdot \left[i + m - n - ik + s + 1 + is - s\right] \\
     &=& 2 \cdot \left[i \cdot (s - k + 1) + m - n + s - s + 1 \right] \\
     &=& 2 \cdot (m - n + 1)
\end{eqnarray*}

Siguiendo un razonamiento en esencia an\'alogo, puede extenderse este caso para valores de $e$ arbitrarios, tomando 
$k = (e-1) \cdot s + 1$. En este escenario,

$$r(n) = e \cdot (m - n + 1)$$

Por supuesto, esta f\'ormula es v\'alida para cualquier $n \leq m$. Si $n > m$, se tiene que $r(n) = 0$.

\subsubsection{Obteniendo una expresi\'on m\'as general}

A partir del resultado anterior y de una observaci\'on cuidadosa de los valores de $r(n)$ para valores de $k$ 
arbitrarios (siempre en el marco de $e = 2$), se conjetur\'o que

$$r(n) = 2 \cdot \ceil{\frac{m - n + 1}{k-s}}$$

A los efectos de probar su validez, notemos primero que $r(n)$ puede definirse recurrentemente:

$$r(n) = r(n+k) + r(M(n+k)) + 2$$

Esta igualdad se deriva instant\'aneamente de la definici\'on de $M$: la cantidad de llamadas recursivas para
computar $M(n)$ son las utilizadas para computar $v = M(n+k)$ sumado a las utilizadas para computar $M(v)$. Por 
cada uno de estos t\'erminos, se suma tambi\'en 1 dado que hay que adicionar la llamada original a $M$ en sendos
casos.\\

Luego, para probar la validez de la conjetura, basta con insertar la f\'ormula en esta expresi\'on y 
comprobar la igualdad. Notemos antes lo siguiente:

$$M(n) = n + \frac{r(n)}{2} \cdot k - \left(1 + \frac{r(n)}{2}\right) \cdot s 
       = n + \left( \ceil{\frac{m-n+1}{k-s}} \right) \cdot k - \left(1 + \ceil{\frac{m-n+1}{k-s}} \right) \cdot s$$

En lo que sigue, notemos $\gamma = \ceil{\frac{m-n-k+1}{k-s}}$. Luego, tenemos:

\begin{itemize}
    \item $r(n+k) = 2 \gamma$
    \item $M(n+k) = n + (\gamma + 1) \cdot k - (\gamma + 1) \cdot s = n + (\gamma + 1) \cdot (k-s)$
    \item $r(M(n+k)) = 2 \cdot \ceil{\frac{m - n - (\gamma + 1) \cdot (k-s) + 1}{k-s}}$
\end{itemize}

En consecuencia,

\begin{eqnarray*}
2 + r(n+k) + r(M(n+k)) &=& 2 + 2 \gamma + 2 \cdot \ceil{\frac{m - n - (\gamma + 1) \cdot (k-s) + 1}{k-s}} \\
                       &=& 2 \cdot \left(1 + \gamma + \ceil{\frac{m - n - (\gamma + 1) \cdot (k-s) + 1}{k-s}} \right) \\
                       &=& 2 \cdot \left(1 + \gamma + \ceil{\frac{m-n+1}{k-s} - (\gamma + 1)} \right) \\
                       &=& 2 \cdot \left(1 + \gamma + \ceil{\frac{m-n+1}{k-s}} - \gamma - 1 \right) \\
                       &=& 2 \cdot \ceil{\frac{m-n+1}{k-s}} \\
                       &=& r(n)
\end{eqnarray*}

Por lo tanto, esto permite derivar una expresi\'on cerrada para $M(n)$ cuando $e = 2$, que adem\'as puede
definirse para cualquier valor de $n \in \Nat$:

$$M(n) = n - s + (k - s) \cdot \text{m\'ax}\left(0, \ceil{\frac{m-n+1}{k-s}}\right)$$

Si instanciamos este resultado para el caso de $M_{91}$, tenemos:

$$\Mc{n} = n - 10 + (11 - 10) \cdot \text{m\'ax}\left(0, \ceil{\frac{100-n+1}{11-10}}\right) = n - 10 + (100 - n + 1) = 91$$

cuando $n \leq 100$. Cuando $n > 100$, se ve tambi\'en que $\Mc{n} = n - 10$.

\end{document}
